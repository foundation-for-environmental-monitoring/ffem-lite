%! Suppress = LineBreak
%! Suppress = MissingLabel
\documentclass{article}
\usepackage[scaled]{helvet}
\renewcommand\familydefault{\sfdefault}
\usepackage[T1]{fontenc}

\setlength{\parindent}{0em}
\setlength{\parskip}{1em}

\title{User Manual for ffem Lite }
\author{ffem}

\begin{document}

    \maketitle
    
       
\section{Important information }

ffem lite is a screening test, and is not a comprehensive test. Confirm the results of this test with tests at a certified laboratory before you take a decision on using the water you are testing.

This test uses chemical reagents. Keep the reagents and other away from children and animals. Do not allow the chemicals to come in contact with your body. Wear gloves and other protective gear to avoid chemical contacts on your body.

 Avoid spillage of reagents.


\section{Introduction}

ffem lite is a simple, open source method of testing water samples. It uses an application which is on a smart phone which measures the amount of a particular parameter present in the given sample when the specified reagents are mixed with the sample in a specified quantity and time.

\section{Procedure}

The water sample which is about to be tested, a specified amount of the sample is mixed with the reagent for the specific parameter that you want to know its concentration. After mixing the solution forms a color or the solution is kept aside for a specified amount of time for the generation of the color in the solution.

After formation of the color the solution, the colored solution is transferred to the cuvette that has been provided with the kit. Download and install ffem lite application from the Google PlayStore.

Set up the stand and take the color card for the parameter that you are working with. Place the color card and the cuvette with the solution on the stand.
 Open the app on the smartphone and tap on the (+) sign at the bottom of the screen. A fading image of the color card appears to guide to hold the smart phone against the color card. Ensure the ambient light is enough and make sure that it is not against the direct sunlight.
  The phone takes a picture,scans it and shows the results.

    
\section{Before you begin}
    
 Before you start, you should have received the reagent pack and the color cards with the QR code on them and a cuvette to put the reaction mixture.
   
    \textbf{Caution} :  All safety measures should be taken. 


\section{Getting help}
Send email to \textbf{support@ffem.io} 

\section{pH}
  
pH is the measure of the acidity of the given solution. Example the pH of the drinking water is around  $6.5$.  pH  is measured in the range of $1-10$ where $1$ being the most acidic and $10$ being the most basic solution.
    
    Materials required for pH testing
    \begin{itemize}
    \item Sample water solution.  
    \item A 10ml measuring tube. 
    \item A smart phone with ffem lite App installed.
    \item An universal pH indicator.
    \item A cuvette and color card for pH.
    \item Tissue paper.
\end{itemize}   

Steps Involved:
  
\begin{enumerate}
\item 10ml of Sample water in the measuring tube.

Rinse the measuring tube with the sample solution.
Take 10 ml of the sample into the measuring tube accurately and close the measuring tube. 
\item Add $5$ drops of the Universal pH reagent.

Open the measuring tube and add exactly $5$ drops of the indicator to the measuring tube with $10$ml of the sample solution and shake the measuring tube well to ensure the mixing of the indicator. Color formed in the measuring tube.
\item Transfer the solution to the cuvette.

Take the cuvette which has been provided to you and add a small amount of the solution from the measuring tube and rinse the cuvette and then add around 3/4th of the solution to the cuvette,set up the stand and also place the color card for pH on the stand now keep the cuvette at the space provided to it on the stand. Careful while placing the cuvette not to spill the solution on the color card.
 \item Open the ffem lite app.
 
 Open the ffem app on your smart phone and click on the  \begin{large}(+)\end{large}sign at the bottom of the screen. Place the phone at the color card as it is shown in the fading graphics of the color card.The phone takes a picture of the color card and the cuvette placed on the stand.
 \item Results  
 
 The pH of the solution will be displayed on your phone.
 
\end{enumerate} 


\section{Fluoride}
    
Fluoride is the measure of the fluoride ion  ($f^-$)present in the given solution.
   
Materials required for testing Fluoride.
  \begin{itemize}
  \item Water sample.
  \item 1 vail of Fluoride reagent.
  \item Measuring tube(10ml)
  \item A smart phone with ffem lite app installed.
  \item  Cuvette and A Color card for Fluoride.
  \item Tissue paper.
  \end{itemize}
  
 \begin{large}
 Steps Involved 
 \end{large}
 
 \begin{enumerate}
 \item Rince the measuring tube with the sample water solution.
 Take 10ml of sample water in the measuring tube.
 \item Add a vial of the fluoride reagent to the measuring tube and mix it well to ensure that there is formation of color.
 \item Set the stand and take the color card for fluoride and place it on the stand 
 \item Take the cuvette and add $\frac{3}{4}$th of the cuvette with the solution from the measuring tube. Before adding rince the cuvette with the solution once and then keep it at the place provided for the cuvette.
 
 \item Open the ffem lite app and tap on the (+) sign at the bottom of the screen on your phone.
 \item Place the phone at the color card as it is shown in the fading graphics of the color card.The phone takes a picture of the color card and the cuvette placed on the stand.
 \item Result 
 
 The phone shows the amount of fluoride present in the given solution.
 
 \end{enumerate}
    


\section{Nitrate}
    
It is the measure of Nitrate ions$(NO_3^-)$ in the given solution.
    
Materials required for testing Nitrate.
  \begin{itemize}
  \item Water sample.
  \item  Nitrate reagent A.
  \item Nitrate reagent B.
  \item Measuring tube(10ml).
  \item A smart phone with ffem lite app installed.
  \item Tissue paper.
  \item  Nitrate color card.
  \item  Cuvette.
  \end{itemize}

 
 \begin{large}
 Steps Involved 
 \end{large}
 
 \begin{enumerate}
 \item Rince the measuring tube with the sample water solution. Take 10ml of sample water in the measuring tube.
 \item Add $4$ drops of Nitrate Reagent A.
 \item Add $4$ drops of Nitrate Reagent B.
 \item Shake well to ensure mixing and wait for $5$minutes for the color generation.
 \item Set up the stand and place the color card for Nitrate. Fill around $\frac{3}{4}$th of the cuvette with  the solution and place the cuvette on the stand at the place provided.
  \item Open the ffem lite app and tap on the (+) sign at the bottom of the screen on your phone.
 \item Place the phone at the color card as it is shown in the fading graphics of the color card.The phone takes a picture of the color card and the cuvette placed on the stand.
 \item Result 
 
 The phone shows the amount of Nitrate present in the given solution.
 \end{enumerate}

\section{Iron}

It is the measure of Fe ions present in the given solution.
    
Materials required for testing Iron.
  \begin{itemize}
  \item Water sample.
  \item Iron Reagent A
  \item Iron Reagent B
  \item Measuring tube(10ml).
  \item A smart phone with ffem lite app installed.
  \item Tissue paper.
  \item  Iron color card
  \item  Cuvette.
  \end{itemize}

 
 \begin{large}
 Steps Involved 
 \end{large}
 
 \begin{enumerate}
 \item Rince the measuring tube with the sample water solution. Take 10ml of sample water in the measuring tube.
 \item Add 0.5ml of Iron Reagent A
 \item Add 2.5ml of Iron Reagent B
 \item Shake well to ensure mixing.
 \item Set up the stand and place the color card for Iron. Fill around $\frac{3}{4}$th of the cuvette with  the solution and place the cuvette on the stand at the place provided.
  \item Open the ffem lite app and tap on the (+) sign at the bottom of the screen on your phone.
 \item Place the phone at the color card as it is shown in the fading graphics of the color card.The phone takes a picture of the color card and the cuvette placed on the stand.
 \item Result 
 
 The phone shows the amount of Iron present in the given solution.
 \end{enumerate}
    
\section{Contributing}

Source code for this method can be found at https://github.com/foundation-for-environmental-monitoring/ffem-lite .


\end{document}
